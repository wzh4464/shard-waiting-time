% @Author: WU Zihan
% @Date:   2022-09-09 10:27:47
% @Last Modified by:   WU Zihan
% @Last Modified time: 2022-09-09 22:54:47

\section*{Queueing Modeling}

The arrival of transactions to the system can be regarded as a Poisson process with parameter \(\lambda\). 
The transactions are divided into two parts, intra-shard transactions and cross-shard transactions. 
For a balanced sharding scheme, the probability of a transaction involving $n$ accounts being an intra-shard transaction can be calculated as 
\[\operatorname{P}_{\rm intra} = \cfrac{1}{k^{n-1}}.\]
Thus, the arrival rate of intra-shard transactions \(\lambda_{\rm int}\) and cross-shard transactions \(\lambda_{\rm cr}\)
can be determined by 
\begin{align*}
    \lambda_{\rm int} &= \lambda \operatorname{P}_{\rm intra} = \cfrac{\lambda}{k^{n-1}}; \\
    \lambda_{\rm cr} &= \lambda \left(1- \operatorname{P}_{\rm intra}\right) = \lambda\left(1-\cfrac{1}{k^{n-1}}\right).
\end{align*}
Suppose that on average, the number of the input accounts is $x$. Then the rate of generating the first-phase subtransactions is
\[\lambda_{\rm sub_1} = x \lambda_{\rm cr} = x \lambda\left(1-\cfrac{1}{k^{n-1}}\right).\]

Now consider an arbitrary shard $i \in \{1,2,\cdots ,k\}$. 
The mining process of shard $i$ is modelled as an $M/M/1$ queue.
Due to the homogeneity assumption of those shards, one can denote the service rate of each shard as $\mu$ and 
the shards share the same arrival rate, denoted by $\lambda_0$.

When the system attains its statistical equilibrium, as shown in \cite{bhat2008IntroductionQueueingTheory}, pp. 42, the
departure process of a $M/M/1$ queue is the same Poisson as the arrival process, from which we can deduct that
the rate of generating the second-phase subtransactions $\lambda_{\rm sub_2}$ satisfies that
\[\lambda_{\rm sub_2} = (n-x) \lambda_{\rm sub_1} = (n-x)x \lambda\left(1-\cfrac{1}{k^{n-1}}\right).\]
Notice that, $\lambda_{\rm sub_2}$ also serves as an input sequence. 
So for shard $i$, along with the homogeneous assumption, we have 

\begin{align*}
    \lambda_0(k) &= \frac{\lambda_{\rm int} + \lambda_{\rm cr}+ \lambda_{\rm sub_2}}{k} \\ &= \frac{1}{k^n}+x(1+n-x)\left(\frac{1}{k}-\frac{1}{k^n}\right).
\end{align*}
Consider the growth of $\lambda_0$, we have
\begin{align*}
    &\lambda_0(k+1)-\lambda_0(k)  \\ &= \cfrac{-B_{n,x}(k+1)^{n-1}k^{n-1}+(B_{n,x}-1)\left[(k+1)^n-k^n\right]}{(k+1)^nk^n} < 0 
\end{align*}
as $k\to \infty$, where $B_{n,x}=x(1+n-x)>0$. This indicates that when $k$ is sufficiently large, $\lambda_0$ will decrease as $k$ gets bigger.


Now we can calculate the expectation of the waiting time $W_q$ as \cite{bhat2008IntroductionQueueingTheory}, pp. 37,
\begin{align*}
    W_q &= \frac{1}{\mu - \lambda_0} - \frac{1}{\mu} \\ 
\end{align*}
If we consider the variation of $W_q$ with $\lambda_0$,
\begin{align*}
    \cfrac{\mathrm{d}W_q}{\mathrm{d}\lambda_0} = \cfrac{1}{(\mu - \lambda_0)^2} > 0
\end{align*}
So minimizing $W_q$ is the same to minimize $\lambda_0$. And since we have known that when $k$ is sufficiently large,
$\lambda_0$ decrease as $k$ tends to infinity, one can conclude that if we want to shorten the waiting time $W_q$, we only need to
increase the number of shards, i.e. integer $k$.